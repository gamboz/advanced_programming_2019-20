\documentclass[a4paper,11pt]{article}
\usepackage{jhep}
\usepackage{wrapfig}
\usepackage{tikz}
%% \usetikzlibrary{trees}

\addtolength{\textwidth}{1in}
\addtolength{\textheight}{1in}
\parindent=0pt
%\setlength{\parskip}{\bigskip}
%\let\parskip\bigskip
%\parskip=\bigskip
\setlength{\parskip}{1ex plus 1ex}

\usepackage{smartdiagram}

\definecolor{mathorange}{HTML}{F68B1F}
\definecolor{mathgreen}{HTML}{2D532A}

\hypersetup{urlcolor=mathgreen}

\usepackage{forest}
\usetikzlibrary{arrows.meta}
\forestset{
  dir tree/.style={
    for tree={
      parent anchor=south west,
      child anchor=west,
      anchor=mid west,
      inner ysep=1pt,
      grow'=0,
      align=left,
      edge path={
        \noexpand\path [draw, \forestoption{edge}] (!u.parent anchor) ++(1em,0) |- (.child anchor)\forestoption{edge label};
      },
      font=\sffamily,
      if n children=0{}{
        delay={
          prepend={[,phantom, calign with current]}
        }
      },
      fit=band,
      before computing xy={
        l=2em
      }
    },
  }
}


\begin{document}
\pagestyle{empty}

\textbf{\color{mathorange}\sffamily\LARGE Advanced Programming 2019/20}

\bigskip

\begin{wrapfigure}{r}{136pt}
  \tiny
\begin{forest}
  dir tree
[ \href{https://github.com/asartori86/advanced_programming_2019-20}{adv. prog. 2019/20}
  [ lectures
    [ c++
      [ 01 intro ]
      [ 02 functions and arrays ]
      [ 03 more on pointers and vectors ]
      [ 04 custom types ]
      [ 05 copy move semantics ]
      [ 06 error handling ]
      [ 07 inheritance ]
      [ 08 STL ]
      [ 09 symbols ]
      [ 10 mixing ]
    ]
    [ python
      [ 01 intro ]
      [ 02 variables and functions ]
      [ 03 advanced topics ]
    ]
  ]
  [ exercises
    [ c++
      [ 01 intro ]
      %% [ \dots ]
      [ 02 arrays ]
      [ 03 arrays and vectors ]
      [ 04 custom types ]
      [ 05 copy move ]
      [ 06 linked list ]
      [ 07 vector ]
      [ 08 generic programming ]
      [ 09 count operations ]
      [ 10 symbols ]
    ]
    [ python
      %% [ \dots ]
      [ 01 intro ]
      [ 02 modules and classes ]
      [ 03 advanced ]
    ]
  ]
  [ exam ]
]
\end{forest}
\end{wrapfigure}

Impressions on the
\href{https://www.math.sissa.it/course/phd-course-master-course/advanced-programming-1}
     {Sissa course} on C++ and Python.

The course covered concepts and syntax of C, C++ and python and how to
mix the three languages, with references to auxiliary tools such as
git and make and best practices, such as never use magic numbers
(always define variables), divide code into simple functions and, in
general, ``think''.

On the right, there is an outline of the git repository used, which
also reflects the course structure.

The course consisted of interactive frontal lessons. I've found it
very useful that the teacher wrote code on the fly in answer to
students' questions or to explore curious case (for instance the use
of templates in headers and the order of linking to produce different
behaviors). We (the students) had access to the same code and could
also edit and tweak what we found difficult.  One lessons every two or
tree was dedicated to supervised exercises.

Other themes have been discussed:
% that go under the umbrella of ``best practices'':\\

\smartdiagramset{
  uniform color list=mathorange!80!white for 4 items,
  description title font=\sffamily\color{mathgreen},
  description title width=11ex}
\smartdiagram[descriptive diagram]{
  {concepts,
    compilation vs.\ linking},
  %
  {make files,
    syntax and tricks},
  %
  {git,
    basic modus operandi},
  %
  {misc.,
    {stl, clang-format, jupyter lab}},
}

\bigskip
What I personally appreciated most was the precision and depth of the
discussion (for instance, we saw some of the differences between c++
11 and 17 or some corner cases of python behavior when mixing mutable
and non-mutable types) and the no non-sense approach to the tools
(git, make \& co.). For instance, we have been invited to fork the
teacher's git repo.\ and work on our copy, merging the new code that
the teacher pushed to his repository before each lesson. This simple
procedure let us gain confidence with git.

Also, I think I finally understood the cause of some compilation and
linking problems that I've met in the years, and, in general, how
shared libraries and object files are structured.
%what the (for me) mystic ``symbols'' are.

\begin{flushright}
\scriptsize  mg Jan.\ 2020
\end{flushright}
\end{document}
